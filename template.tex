\documentclass[10pt]{beamer}

%\usetheme[progressbar=frametitle]{metropolis}
\usetheme{jeanmichelbeamer}

\title{Beamer template}
\subtitle{Basic theme}
\date{\today}
\author{Mathieu Carrié}
\institute{Prout}

\begin{document}

\maketitle

\begin{frame}[plain]{Table of contents}
  \setbeamertemplate{section in toc}[sections numbered]
  \tableofcontents[hideallsubsections]
\end{frame}

\section{Usage}

\begin{frame}[fragile]{Usage}
    JeanMichel Beamer is a simple theme.
    Use the theme as any other beamer theme:

    \begin{verbatim} \usetheme{jeanmichelbeamer} \end{verbatim}

    The theme can also be used as selectively: \texttt{outer}, \texttt{inner}, \texttt{color} and \texttt{font} themes are separated.
    This theme is a very minimalist one inspired from Metropolis.
\end{frame}

\begin{frame}{Dependencies}
    \begin{itemize}
        \item XeTeX or LuaTex for font related issues
        \item Regular packages included in most TeX distributions
    \end{itemize}
\end{frame}

\section{Features}

\begin{frame}{Sections frames}
    Section frames are created to ease transitions between sections. Note that a progressbar is displayed on these frames and at the bottom of regular ones.
\end{frame}

\begin{standout}
    Simple standout frames
\end{standout}

\section{Other section}

\begin{frame}{Special blocks}
  \begin{block}{Default}
    Block content.
  \end{block}

  \begin{alertblock}{Alert}
    Block content.
  \end{alertblock}

  \begin{exampleblock}{Example}
    Block content.
  \end{exampleblock}
\end{frame}

\end{document}
